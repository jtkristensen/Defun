In the article ``{\it Defunctionalization, a denotational investigation}'',
Lasse R. Nielsen and Olivier Danvy prove that the naive type-based transformation
is semantically preserving. And in ``{\it Defunctionalization at Work}'',
the same authors, the practical applications of the transformation
are investigated.
Moreover,
Ryan Shea and Matthew Fluet, compare state of the art defunctionalization algorithms
in the article
``{\it Alternate Control-Flow Analyses for Defunctionalization in the MLton Compiler}''
.
Besides these projects, I find that the related work fall into one of two categories.
There are some Phd'thesiss where people use hundreds of pages in formalizing the
theoretical properties of the transformation when more complicated analysis
is used, when applied to programs with polymorphism and so on.
The other category is closer to the original article by Reynolds,
where the transformation is performed ad-hoc but not really the focus of the work.
